
\documentclass[manuscript]{aastex62}

\usepackage{hyperref}
\usepackage[caption=false]{subfig}
\usepackage{listings}
\usepackage{amsmath}
\usepackage{graphicx}
\usepackage{float}
\lstset{
basicstyle=\small\ttfamily,
columns=flexible,
breaklines=true
}

\begin{document}


\title{Exercises in Mainline Stellar Nuclear Burning}

\shorttitle{Exercises in Mainline Stellar Nuclear Burning}

\author{Ansh Sehgal}
\affil{Department of Physics and Astronomy, Clemson University, Clemson, SC 29634-0978}

\begin{abstract}
We examine the life cycle of a star by decomposing its nucleosynthesis into distinct burning stages, or "shells" to better understand the underlying processes in stellar interiors. We run calculations at increasing temperatures and densities, until the formation of metallic elements like Iron, signaling the exhaustion of the initial stellar nuclear fuel.
\end{abstract}

\section{Introduction}

Stellar nucleosynthesis follows a fairly well-understood sequence of burning
stages.  These are reviewed in a several text books, including
\cite{1983psen.book.....C,1996snai.book.....A,2007nps..book.....I}.
The purpose of this brief paper is to present a set of computational
exercises using
\href{http://sourceforge.net/projects/nucnet-tools/}{NucNet Tools}
to demonstrate these stages in a massive star.

Together we will walk through the basic stages of element fusion inside of stars, from Hydrogen and Helium to the more metallic ones such as Iron and Nickel.

\section{Methods} \label{sec:docker}

We use the docker repository for the nucleosynthesis code found in :

\href{http://github.com/mbradle/docker_single_zone}{docker single-zone}


To run, use a command like:
\begin{lstlisting}
docker run -it -v ${PWD}/work/input:/input_directory -v ${PWD}/work/output:/output_directory -e VAR="@run_h.rsp" single_zone:default
\end{lstlisting}
where {\it run\_h.rsp} reads
\begin{verbatim}
--t9_0 0.03 --rho_0 50.
--network_xml /input_directory/data_pub/my_net.xml
--zone_xml /input_directory/data_pub/Lodders.xml
--nuc_xpath "[z <= 94]" --iterative_solver gmres --iterative_t9 2
--output_xml /output_directory/out.xml --tend 3.15e15
\end{verbatim}

But we will edit the response file to update the increasing temperature and density at every new burning stage.

\section{Hydrogen Burning} \label{sec:H}

The first stellar burning phase is hydrogen burning, which converts
abundant hydrogen into helium.  In a massive star
this occurs at higher temperature but lower density than in the Sun.


Plotted below is the mass fraction of the Hydrogen and resulting Helium that is generated due to the pp chain reaction:


\begin{figure}[H]
\centering
\includegraphics[scale=0.7]{task1}
\caption{log plot of $^1$H and $^4$He as a function of time}
\end{figure}


We can also examine the abundances of the Carbon, Neon and Oxygen isotopes during the CNO cycle:

\begin{figure}[H]
\centering
\includegraphics[scale=0.7]{task1_1}
\caption{log plot of $^{12}$C,$^{14}$N, and $^{16}$O as a function of time}
\end{figure}

As we see, there is a buildup of $^{14}$N which can be attributed to the rather slow reaction time of the process, which comes after. Namely:

\begin{equation}
^{14}N \rightarrow ^{15}O + \gamma
\end{equation}

Because this reaction is so slow, there is a natural saturation of Nitrogen that occurs. 

\section{Helium Burning} \label{sec:He}



Once a star burns hydrogen to helium, it contracts and heats until the He
ignites.  The He burns to C and O (but not much Ne as we see below)


\begin{figure}[H]
\centering
\includegraphics[scale=0.7]{task2}
\caption{log plot of $^4$He,$^{12}$C, $^{16}$O as a function of time. $^{20]}$Ne is also plotted to show that Oxygen is really the stopping point for the triple alpha process}
\end{figure}


In this triple-$\alpha$ process, the $^4$He 's combine together until they form Carbon and Oxygen, leading to the eventual buildup of the $^{16}$O seen in the plot. However, $^{20}$Ne is not formed as much, due to the fact that the energy needed to overcome the stability of the Oxygen, and the subsequent Coulomb barrier is too much. Only at extreme temperatures (closer to the core) do these elements start to form. 

We now wish to consider the s-process ("slow") in Helium burning. 

\begin{figure}[H]
\centering
\includegraphics[scale=0.7]{task3}
\caption{log plot of $^{14}$N,$^{18}$O, $^{22}$Ne as a function of time.}
\end{figure}

The Nitrogen slowly combines with alpha particles to form $^{18}$O, which can then combine with more Helium to form the $^{22}$Ne.

\begin{figure}[H]
\centering
\includegraphics[scale=0.7]{neutron}
\caption{Neutron Number Density as a function of time, on a log-log scale}
\end{figure}

Here we plot the number density of neutrons over time. In particular, there is a short burst from the reaction:

\begin{equation}
^{13}C + ^4He \rightarrow ^{16}O + n
\end{equation}

The products include a neutron, leading to the increase of $n_n$. Then, there is a more steady supply once the $^{22}$Ne has formed as mentioned above.

Finally, we examine the abundances of each element at the beginning and end of the calculation to see the evolution of the composition of the star. 

\begin{figure}[H]
\centering
\includegraphics[scale=0.7]{task3_2}
\end{figure}

Notice how the final calculation yields higher abundances of heavier elements, which is to be expected given that the hydrogen and helium is being depleted to fuse more metallic elements. The graph spikes right around Carbon and Oxygen, as we expect.

\section{Carbon Burning} \label{sec:C}



The principal products of helium burning are $^{12}$C and $^{16}$O.  As the
star contracts and heats, carbon burning begins.

The key reactions that take place are:

\begin{align}
^{12}C + ^{12}C \rightarrow ^{24}Mg^* \rightarrow ^{24}Mg + \gamma \\
\rightarrow ^{20}Ne + ^{4}He \\
\rightarrow ^{23}Na + p \\
^{12}C +^{16}O \rightarrow ^{24}Mg + ^{4}He
\end{align}


\begin{figure}[H]
\centering
\includegraphics[scale=0.7]{task4}
\caption{Mass Fraction of the various relevant isotopes to Carbon Burning as a function of time}
\end{figure}

Notice that as the $^{12}C$ and $^{16}O$ is used up in the aforementioned reaction, there is a commensurate rise in the $^{20}Ne$ and $^{24}Mg$

Also note that if the star forms a white dwarf after this burning,
it will be a O/Ne/Mg white dwarf, not a C/O white dwarf, as it would have
been after helium burning.

\section{Neon Burning} \label{sec:Ne}

Following carbon burning, the star will contract, heat, and undergo neon
burning.  This burning stage comprises $^{20}{\rm Ne} + \gamma \to ^{16}{\rm O}
+ ^4{\rm He}$ and $^{20}{\rm Ne} + ^4{\rm He} \to ^{24}{\rm Mg} + \gamma$,
which effectively is $^{20}{\rm Ne} + ^{20}{\rm Ne} \to ^{16}{\rm O} +
^{24}{\rm Mg}$.

We run the calculation for 1000 years at an increased $\rho$ and T$_9$
and find the following abundances

\begin{figure}[H]
\centering
\includegraphics[scale=0.7]{task5}
\caption{Mass Fraction of the various relevant isotopes to Neon Burning as a function of time}
\end{figure}


These match our predictions from the reaction equations above; in particular, the rise of $^{16}O$  and subsequent depletion of $^{20}Ne$ 

\section{Oxygen Burning} \label{sec:O}

After neon burning, the next stage is oxygen burning.  The principal reaction
is $^{16}{\rm O} + ^{16}{\rm O} \to ^{32}{\rm S}^*$.  The excited $^{32}$S
nucleus then de-excites mostly to $^{28}$Si (and an alpha)
or $^{32}$S.

This is represented in our mass fractions:

\begin{figure}[H]
\centering
\includegraphics[scale=0.7]{task6}
\caption{Mass Fraction of the various relevant isotopes to Oxygen Burning as a function of time}
\end{figure}

We run the calculation for a $T_9 = 2.1$ and $\rho = 1.5\times 10^6$ g/cc for one year.

Also occurring during oxygen burning is the ``gamma'' process.  The initially
present heavy nuclei are disintegrated.  First neutrons then protons and
alphas are emitted.  The nuclei ``flow'' down the proton-rich side of the
stable nuclei.  If this matter ``freezes out'', proton-rich nuclei are
formed (the so-called ``p-nuclei'').


We take a look at the overall abundances as a function of mass number between the initial and final calculation. The overplotting gives a sense of the time evolution.

\begin{figure}[H]
\centering
\includegraphics[scale=0.7]{task7}
\caption{Abundances versus corresponding atomic number, at both the initial and final calculation time step}
\end{figure}

Note how nuclei ``melt'' towards the
iron-group region. 

\section{Silicon Burning} \label{sec:Si}

The final mainline burning stage is Si burning.  Here $^{28}$Si converts to
$^{56}$Ni, which decays to $^{56}$Fe.  The burning proceeds through a
complicated sequence--some Si disintegrates and the light particles that result
capture onto the remaining Si to create heavier nuclei.  In fact the
burning proceeds through a quasi-equilibrium
\cite{1968ApJS...16..299B,1998ApJ...498..808M}.


\begin{figure}[H]
\centering
\includegraphics[scale=0.7]{task8}
\caption{Mass Fraction of the various relevant isotopes to Silicon Burning as a function of time}
\end{figure}
Now the star has contracted to an extreme level, with $T_9 = 3.5$ and $\rho = 1.5e7$ g/cc in our 1 year calculation.


You might be surprised that the abundance of $^{56}$Ni is never particularly
large.  The reason is that the silicon burning is sufficiently slow that
weak decays have time to occur during the burning and thus change the
neutron richness of the nuclei present.  Also, at this temperature, there
is a preference for $^{54}$Fe and two protons over $^{56}$Ni, as evidence by its relatively high abundance. 

\section{Conclusion}

Of course stellar burning really occurs in the context of an evolving star.
The temperature and density are not constant, as treated in these exercises,
but rather evolve as nuclear fuel is consumed.  Furthermore, in many cases,
stellar layers are convective, which tends to mix the composition of those
layers during the burning.  
Nevertheless, we hope that by
carrying out the above exercises, one will get a good sense of the
nature of the mainline stellar burning phases.

\begin{thebibliography}{5}
\expandafter\ifx\csname natexlab\endcsname\relax\def\natexlab#1{#1}\fi

\bibitem[{{Arnett}(1996)}]{1996snai.book.....A}
{Arnett}, D. 1996, {Supernovae and nucleosynthesis. an investigation of the
  history of matter, from the Big Bang to the present} (Princeton series in
  astrophysics, Princeton, NJ: Princeton University Press, |c1996)

\bibitem[{{Bodansky} {et~al.}(1968){Bodansky}, {Clayton}, \&
  {Fowler}}]{1968ApJS...16..299B}
{Bodansky}, D., {Clayton}, D.~D., \& {Fowler}, W.~A. 1968, \apjs, 16, 299

\bibitem[{{Clayton}(1983)}]{1983psen.book.....C}
{Clayton}, D.~D. 1983, {Principles of stellar evolution and nucleosynthesis}
  (Chicago: University of Chicago Press, 1983)

\bibitem[{{Iliadis}(2007)}]{2007nps..book.....I}
{Iliadis}, C. 2007, {Nuclear Physics of Stars} (Wiley-VCH Verlag)

\bibitem[{{Meyer} {et~al.}(1998){Meyer}, {Krishnan}, \&
  {Clayton}}]{1998ApJ...498..808M}
{Meyer}, B.~S., {Krishnan}, T.~D., \& {Clayton}, D.~D. 1998, \apj, 498, 808

\end{thebibliography}
\end{document}
